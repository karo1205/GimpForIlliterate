\scene[. -- Begr��ung \& Einf�hrung]

\StageDir{
\begin{center}
Folie Nummer 1
\end{center}
}


\begin{drama}
\bernhardspeaks Ich darf euch ganz herzlich zum zweiten Vortrag der Vortragsreihe zum Thema Gimp begr��en. F�r all jene, die beim ersten Vortrag nicht anwesend waren, m�chte ich an dieser Stelle nochmals erw�hnen, dass diese Vortragsreihe parallel zum Grundkurs abgehalten wird. Selbstverst�ndlich sind wir uns dar�ber im Klaren, dass der Grundkurs noch nicht sehr weit fortgeschritten ist und hier vielleicht W�rter verwendet werden, die etwas komplizierter sind und im Grundkurs noch nicht behandelt wurden. Falls ihr W�rter nicht versteht, dann habt keine Scheu und unterbrecht uns einfach ganz kurz, das ist kein Problem. Obwohl in Gimp viele Funktionen durch sch�ne kleine Bildchen veranschaulicht werden, gibt es ein paar Funktionen die man nur durch das Men� erreichen kann. Am Anfang kann es vorkommen, dass ihr hier einzelne Funktionen vertauscht, da sich die W�rter mitunter doch sehr �hneln. Das ist v�llig normal und kein Problem, doch mit voranschreiten des Grundkurses wird euch die Bedienung von Gimp immer einfacher fallen. \newline \newline Soweit zu unserem kleinen Beispiel. Und damit wieder zu Niko.
\end{drama}

\scene[. -- R�ckblick auf den Inhalt des ersten Vortrages]


\StageDir{
\begin{center}
Folie mit den Werkzeugen
\end{center} 
}

\begin{drama}
\bernhardspeaks Wie sich sicher noch einige von euch erinnern k�nnen, haben wir uns im ersten Vortrag mit der Oberfl�che von Gimp vertraut gemacht und uns ein bisschen mit den unterschiedlichen Werkzeugen auseinander gesetzt. F�r all jene von euch, die beim letzten Mal nicht anwesend waren und denen der Begriff Gimp jetzt nichts sagt: Gimp kann man im Prinzip mit Photoshop vergleichen, das sicherlich vielen von euch ein Begriff ist. Im Gegensatz zu Photoshop muss man jedoch f�r Gimp nichts zahlen und kann es gratis aus dem Internet herunterladen. Das ist auch einer der Gr�nde, warum wir uns in dieser Vortragsreihe auf Gimp konzentrieren. \newline \newline Wie bereits erw�hnt, haben wir uns das letzte Mal mit den Werkzeugen auseinandergesetzt. Wir haben unter anderem gelernt, wie man Dinge ausschneiden und wieder einf�gen kann sowie wie man mit den Zeichenwerkzeugen in Gimp umgeht. Wie ihr euch sicher noch daran erinnern k�nnt, haben wir dabei immer auf einer Art "`virtuellem Papier"' gearbeitet. Heute werden wir einen Schritt weiter gehen und euch das Konzept der Ebenen vorstellen. Wir werden zu Beginn einmal kl�ren was denn Ebenen eigentlich sind, wof�r man sie in Gimp verwendet und worin der gro�e Vorteil von Ebenen liegt. Anschlie�end werden wir anhand ein paar einfacher Beispiele zeigen, wie man mit den Werkzeugen auf unterschiedlichen Ebenen arbeiten kann. Und damit geht's auch schon los. Bitte, Niko.
\end{drama}