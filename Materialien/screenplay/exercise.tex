\scene[. -- Live Demo (Pinsel vor K�fig stellen)]

\StageDir{
\begin{center}
Robert stellt Frage ans Publikum: "`Wie ihr seht, ragt die Spitze des Pinsels unter dem K�fig hindurch. Wir w�rden nun gerne das Ganze so �ndern, dass die Spitze durch die Gitterst�be schaut. Wie k�nnen wir das machen?"' \newline \newline
Hier �bernehme dann ich.
\end{center}
}

\begin{drama}
\bernhardspeaks Wir wissen ja, dass wir Ebenen in Gimp rauf - und runterschieben k�nnen. Dadurch werden die Ebenen entweder weiter in den Vordergrund oder weiter in den Hintergrund verschoben. Wir k�nnten jetzt nun auf die Idee kommen die Ebene mit unserem Gimp Maskottchen in den Vordergrund zu verschieben, damit die Pinselspitze nicht mehr unterhalb des K�figs hindurchragt. Doch was wird dann passieren? \newline \newline Richtig, da sich das Maskottchen zusammen mit der Pinselspitze auf einer Ebene befindet, befreien wir das Maskottchen quasi aus dem K�fig, wenn wir die Ebenenreihenfolge �ndern. Die L�sung ist, dass wir die Pinselspitze ausschneiden und auf eine einzelne Ebene wieder einf�gen. Doch wie machen wir das am Besten? \newline \newline Wie ihr euch sicherlich noch erinnern k�nnt, gibt es das Zauberstab Werkezug, mit dem man recht einfach zusammenh�ngende Bereiche ausw�hlen kann, die �hnliche Farben haben. Wenn man mit dem Zauberstab Bereiche ausw�hlt, kann es vorkommen, dass es sinnvoll ist den Schwellwert zu erh�hen. Mit dem Schwellwert legt man fest um wieviel sich benachbarte Farben unterscheiden d�rfen, damit der Bereich als zusammenh�ngend erkannt wird. Nachdem die Pinselspitze ausgew�hlt ist, wird der ausgew�hlte Bereich mit Strg + X ausgeschnitten. Robert ist so nett und schreibt euch das Tastenk�rzel zum Ausschneiden auf die Tafel. \newline \newline Um die ausgeschnittene Pinselspitze wieder einzuf�gen, m�ssen wir eine neue Ebene erstellen. Als n�chstes m�sst ihr sicherstellen, dass die neu erstellte Ebene ausgew�hlt ist, bevor ihr die Pinselspitze wieder mit Strg + V einf�gt. Auch das Tastenk�rzel findet ihr auf der Tafel. Wenn man in Gimp etwas einf�gt, dann erstellt Gimp standardm��ig eine schwebende Auswahl. Die schwebende Auswahl kann man sich ein bisschen wie eine Zwischenebene vorstellen. Damit aus der schwebenden Auswahl eine richtige Ebene wird, muss die Auswahl verankert werden. Das geschieht mit dem Anker-Symbol. Wenn ihr die neue Ebene dann noch in den Vordergrund verschiebt, dann seht ihr wie die Pinselspitze nun aus dem K�fig schaut. Mit dem Augen-Symbol k�nnen wir die neue Ebene ein - und ausblenden und sehen, dass sich die Pinselspitze tats�chlich auf einer eigenen Ebene befindet. 
\end{drama}