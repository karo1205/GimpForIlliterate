\documentclass[a4paper,showtrims,11pt]{memoir}

\usepackage{dramatist}
\usepackage[latin1]{inputenc}


\pagestyle{plain}


\begin{document}



\Character[PERSON A, Person A]{Robert}{robert}
\Character[PERSON B, Person B]{Niko}{niko}
\Character[PERSON B, Person B]{Bernhard}{bernhard}


\scene[. -- Begr��ung \& Einf�hrung]

\StageDir{
\begin{center}
Prezi anzeigen, Folie Nummer 1
\end{center}
}

\begin{drama}
\bernhardspeaks Ich darf euch ganz herzlich zum zweiten Vortrag der Vortragsreihe zum Thema Gimp begr��en. F�r all jene, die beim ersten Vortrag nicht anwesend waren, m�chte ich an dieser Stelle nochmals erw�hnen, dass diese Vortragsreihe parallel zum Grundkurs abgehalten wird. Selbstverst�ndlich sind wir uns dar�ber im Klaren, dass der Grundkurs noch nicht sehr weit fortgeschritten ist und hier vielleicht W�rter verwendet werden, die etwas komplizierter sind und im Grundkurs noch nicht behandelt wurden. Falls ihr W�rter nicht versteht, dann habt keine Scheu und unterbrecht uns einfach ganz kurz, das ist kein Problem. Obwohl in Gimp viele Funktionen durch sch�ne kleine Bildchen veranschaulicht werden, gibt es ein paar Funktionen die man nur durch das Men� erreichen kann. Am Anfang kann es vorkommen, dass ihr hier einzelne Funktionen vertauscht, da sich die W�rter mitunter doch sehr �hneln. Das ist v�llig normal und kein Problem, doch mit voranschreiten des Grundkurses wird euch die Bedienung von Gimp immer einfacher fallen. \newline \newline Soweit zu unserem kleinen Beispiel. Und damit wieder zu Niko.
\end{drama}

\scene[. -- R�ckblick auf den Inhalt des ersten Vortrages]


\StageDir{
\begin{center}
Folie mit den Werkzeugen
\end{center} 
}

\begin{drama}
\bernhardspeaks Wie sich sicher noch einige von euch erinnern k�nnen, haben wir uns im ersten Vortrag mit der Oberfl�che von Gimp vertraut gemacht und uns ein bisschen mit den unterschiedlichen Werkzeugen auseinander gesetzt. F�r all jene von euch, die beim letzten Mal nicht anwesend waren und denen der Begriff Gimp jetzt nichts sagt: Gimp kann man im Prinzip mit Photoshop vergleichen, das sicherlich vielen von euch ein Begriff ist. Im Gegensatz zu Photoshop muss man jedoch f�r Gimp nichts zahlen und kann es gratis aus dem Internet herunterladen. Das ist auch einer der Gr�nde, warum wir uns in dieser Vortragsreihe auf Gimp konzentrieren. \newline \newline Wie bereits erw�hnt, haben wir uns das letzte Mal mit den Werkzeugen auseinandergesetzt. Wir haben unter anderem gelernt, wie man Dinge ausschneiden und wieder einf�gen kann sowie wie man mit den Zeichenwerkzeugen in Gimp umgeht. Wie ihr euch sicher noch daran erinnern k�nnt, haben wir dabei immer auf einer Art "`virtuellem Papier"' gearbeitet. Heute werden wir einen Schritt weiter gehen und euch das Konzept der Ebenen vorstellen. Wir werden zu Beginn einmal kl�ren was denn Ebenen eigentlich sind, wof�r man sie in Gimp verwendet und worin der gro�e Vorteil von Ebenen liegt. Anschlie�end werden wir anhand ein paar einfacher Beispiele zeigen, wie man mit den Werkzeugen auf unterschiedlichen Ebenen arbeiten kann. Und damit geht's auch schon los. Bitte, Niko.
\end{drama} %Begruessung & Einfuehrung

\scene[. -- Live Demo (Pinsel vor K�fig stellen)]

\StageDir{
\begin{center}
Robert stellt Frage ans Publikum: "`Wie ihr seht, ragt die Spitze des Pinsels unter dem K�fig hindurch. Wir w�rden nun gerne das Ganze so �ndern, dass die Spitze durch die Gitterst�be schaut. Wie k�nnen wir das machen?"' \newline \newline
Hier �bernehme dann ich.
\end{center}
}

\begin{drama}
\bernhardspeaks Wir wissen ja, dass wir Ebenen in Gimp rauf - und runterschieben k�nnen. Dadurch werden die Ebenen entweder weiter in den Vordergrund oder weiter in den Hintergrund verschoben. Wir k�nnten jetzt nun auf die Idee kommen die Ebene mit unserem Gimp Maskottchen in den Vordergrund zu verschieben, damit die Pinselspitze nicht mehr unterhalb des K�figs hindurchragt. Doch was wird dann passieren? \newline \newline Richtig, da sich das Maskottchen zusammen mit der Pinselspitze auf einer Ebene befindet, befreien wir das Maskottchen quasi aus dem K�fig, wenn wir die Ebenenreihenfolge �ndern. Die L�sung ist, dass wir die Pinselspitze ausschneiden und auf eine einzelne Ebene wieder einf�gen. Doch wie machen wir das am Besten? \newline \newline Wie ihr euch sicherlich noch erinnern k�nnt, gibt es das Zauberstab Werkezug, mit dem man recht einfach zusammenh�ngende Bereiche ausw�hlen kann, die �hnliche Farben haben. Wenn man mit dem Zauberstab Bereiche ausw�hlt, kann es vorkommen, dass es sinnvoll ist den Schwellwert zu erh�hen. Mit dem Schwellwert legt man fest um wieviel sich benachbarte Farben unterscheiden d�rfen, damit der Bereich als zusammenh�ngend erkannt wird. Nachdem die Pinselspitze ausgew�hlt ist, wird der ausgew�hlte Bereich mit Strg + X ausgeschnitten. Robert ist so nett und schreibt euch das Tastenk�rzel zum Ausschneiden auf die Tafel. \newline \newline Um die ausgeschnittene Pinselspitze wieder einzuf�gen, m�ssen wir eine neue Ebene erstellen. Als n�chstes m�sst ihr sicherstellen, dass die neu erstellte Ebene ausgew�hlt ist, bevor ihr die Pinselspitze wieder mit Strg + V einf�gt. Auch das Tastenk�rzel findet ihr auf der Tafel. Wenn man in Gimp etwas einf�gt, dann erstellt Gimp standardm��ig eine schwebende Auswahl. Die schwebende Auswahl kann man sich ein bisschen wie eine Zwischenebene vorstellen. Damit aus der schwebenden Auswahl eine richtige Ebene wird, muss die Auswahl verankert werden. Das geschieht mit dem Anker-Symbol. Wenn ihr die neue Ebene dann noch in den Vordergrund verschiebt, dann seht ihr wie die Pinselspitze nun aus dem K�fig schaut. Mit dem Augen-Symbol k�nnen wir die neue Ebene ein - und ausblenden und sehen, dass sich die Pinselspitze tats�chlich auf einer eigenen Ebene befindet. 
\end{drama} % Gimp Live Demo (2.Teil - Pinselspitze freistellen)







%%%%%%%%%%%%%%%%%%%%%%%%%%%%%%%%%%%%%%%%%%%%%%%%%%%%%%%%%%%%%%%%%%%%%%%%%%%%%%%%%%%%%%%%%%%%%%%%%%%%%%%%%%%%%%%%%%%

%\scene[. -- Vorwort]

%\StageDir{
%	\begin{center}
%		Keine Folie
%	\end{center}
%}

%\begin{drama}
%\aspeaks Bevor wir mit dem eigentlichen Vortrag starten, noch ein paar erkl�rende Worte zu unserer Zielgruppe. Der Vortrag richtet sich im Wesentlichen an Analphabeten, die sich mit dem Lesen und Schreiben schwer tun. Da der gr��te Teil der Analphabeten in �sterreich zur Gruppe der funktionalen Analphabeten geh�rt - einfache W�rter (z.B der eigene Name) k�nnen geschrieben und kurze S�tze gelesen werden, aber der Sinn l�ngerer Texte wird oftmals nicht verstanden - richtet sich der Vortrag verst�rkt an diese Personen. Das Bemerkenswerte ist, dass viele Analphabeten ihre Schreib - und Leseschw�che sehr gut vor anderen verbergen k�nnen, indem sie zum Beispiel Dinge, die andere aufschreiben w�rden, einfach ausw�ndig lernen (hier k�nnte man die Geschichte der Frau erz�hlen, die am Serviceschalter Auskunft �ber Zugverbindungen gibt und dabei alle Ankunfts - und Abfahrtszeiten auswendig gelernt hat). \newline Soviel zu unserer Zielgruppe. Wir hoffen, dass f�r euch der nachfolgende Vortrag nun etwas verst�ndlicher ist. 
%\end{drama}







%\scene[. -- R�ckblick auf den Inhalt des ersten Vortrages]


%\StageDir{
%\begin{center}
%Folie mit den Werkzeugen
%\end{center} 
%}

%\begin{drama}
%\aspeaks Wie sich sicher noch einige von euch erinnern k�nnen, haben wir uns im ersten Vortrag mit der Oberfl�che von Gimp vertraut gemacht und uns ein bisschen mit den unterschiedlichen Werkzeugen auseinander gesetzt. Unser heutiger Fokus liegt auf den Ebenen. Wir werden zu Beginn einmal kl�ren was denn Ebenen eigentlich sind, wof�r man sie in Gimp verwendet und worin der gro�e Vorteil liegt. Anschlie�end werden wir anhand ein paar einfacher Beispiele zeigen, wie man mit den Werkzeugen auf unterschiedlichen Ebenen arbeiten kann.  
%\end{drama}


%\scene[. -- Ebenen]


%\StageDir{
%\begin{center}
%Folie ????
%\end{center} 
%}

%\begin{drama}
%\bspeaks Nun zu den Ebenen. \textit{Hat von euch vielleicht jemand eine Vorstellung was man sich unter Ebenen vorstellen kann?} Eine Ebene kann man sich im Prinzip wie ein Blatt Papier oder eben diese Tafel vorstellen. An der Tafel kann ich ganz normal zeichnen. Ich kann Dinge hinzuf�gen und Dinge wieder wegl�schen. 
%\end{drama}

%%%%%%%%%%%%%%%%%%%%%%%%%%%%%%%%%%%%%%%%%%%%%%%%%%%%%%%%%%%%%%%%%%%%%%%%%%%%%%%%%%%%%%%%%%%%%%%%%%%%%%%%%%%%%%%%%%%


\end{document}
