\documentclass[a5paper,showtrims,11pt]{memoir}

\usepackage{dramatist}
\usepackage[latin1]{inputenc}

%% Layout
\settrimmedsize{18,5cm}{13cm}{*}
\setlength{\trimedge}{\stockwidth}
\addtolength{\trimedge}{-\paperwidth}
\settrims{0pt}{\trimedge}
\settypeblocksize{*}{22pc}{1.71}
\setlrmargins{*}{*}{1.5}
\setulmargins{*}{*}{1}
\setlength{\footskip}{20pt}
\checkandfixthelayout

\ifpdf
\setlength{\pdfpageheight}{\stockheight}
\setlength{\pdfpagewidth}{\stockwidth}
\fi
\renewcommand{\printscenenum}{%
    \scenenumfont \thescene}
\setlength{\beforesceneskip}{20pt}
\pagestyle{plain}


\begin{document}

\Character[PERSON A, Person A]{PERSON A}{a}
\Character[PERSON B, Person B]{PERSON B}{b}


\scene[. -- Vorwort]

\StageDir{
	\begin{center}
		Keine Folie
	\end{center}
}

\begin{drama}
\aspeaks Bevor wir mit dem eigentlichen Vortrag starten, noch ein paar erkl�rende Worte zu unserer Zielgruppe. Der Vortrag richtet sich im Wesentlichen an Analphabeten, die sich mit dem Lesen und Schreiben schwer tun. Da der gr��te Teil der Analphabeten in �sterreich zur Gruppe der funktionalen Analphabeten geh�rt - einfache W�rter (z.B der eigene Name) k�nnen geschrieben und kurze S�tze gelesen werden, aber der Sinn l�ngerer Texte wird oftmals nicht verstanden - richtet sich der Vortrag verst�rkt an diese Personen. Das Bemerkenswerte ist, dass viele Analphabeten ihre Schreib - und Leseschw�che sehr gut vor anderen verbergen k�nnen, indem sie zum Beispiel Dinge, die andere aufschreiben w�rden, einfach ausw�ndig lernen (hier k�nnte man die Geschichte der Frau erz�hlen, die am Serviceschalter Auskunft �ber Zugverbindungen gibt und dabei alle Ankunfts - und Abfahrtszeiten auswendig gelernt hat). \newline Soviel zu unserer Zielgruppe. Wir hoffen, dass f�r euch der nachfolgende Vortrag nun etwas verst�ndlicher ist. 
\end{drama}




\scene[. -- Begr��ung \& Einf�hrung]

\StageDir{
\begin{center}
Folie Nummer 1
\end{center}
}


\begin{drama}
\bspeaks Ich darf euch ganz herzlich zum zweiten Vortrag der Vortragsreihe zum Thema Gimp begr��en. F�r all jene, die beim ersten Vortrag nicht anwesend waren, m�chte ich an dieser Stelle nochmals erw�hnen, dass diese Vortragsreihe parallel zum Grundkurs abgehalten wird. Selbstverst�ndlich sind wir uns dar�ber im Klaren, dass der Grundkurs noch nicht sehr weit fortgeschritten ist und hier vielleicht W�rter verwendet werden, die etwas komplizierter sind und im Grundkurs noch nicht behandelt wurden. Falls ihr W�rter nicht versteht, dann habt keine Scheu und unterbrecht uns einfach ganz kurz, das ist kein Problem. Obwohl in Gimp viele Funktionen durch sch�ne kleine Bildchen veranschaulicht werden, gibt es ein paar Funktionen die man nur durch das Men� erreichen kann. Am Anfang kann es vorkommen, dass ihr hier einzelne Funktionen vertauscht, da sich die W�rter mitunter doch sehr �hneln. Das ist v�llig normal und kein Problem, doch mit voranschreiten des Grundkurses wird euch die Bedienung von Gimp immer einfacher fallen. 
\end{drama}



\scene[. -- R�ckblick auf den Inhalt des ersten Vortrages]


\StageDir{
\begin{center}
Folie mit den Werkzeugen
\end{center} 
}

\begin{drama}
\aspeaks Wie sich sicher noch einige von euch erinnern k�nnen, haben wir uns im ersten Vortrag mit der Oberfl�che von Gimp vertraut gemacht und uns ein bisschen mit den unterschiedlichen Werkzeugen auseinander gesetzt. Unser heutiger Fokus liegt auf den Ebenen. Wir werden zu Beginn einmal kl�ren was denn Ebenen eigentlich sind, wof�r man sie in Gimp verwendet und worin der gro�e Vorteil liegt. Anschlie�end werden wir anhand ein paar einfacher Beispiele zeigen, wie man mit den Werkzeugen auf unterschiedlichen Ebenen arbeiten kann.  
\end{drama}



\scene[. -- Ebenen]


\StageDir{
\begin{center}
Folie ????
\end{center} 
}

\begin{drama}
\bspeaks Nun zu den Ebenen. \textit{Hat von euch vielleicht jemand eine Vorstellung was man sich unter Ebenen vorstellen kann?} Eine Ebene kann man sich im Prinzip wie ein Blatt Papier oder eben diese Tafel vorstellen. An der Tafel kann ich ganz normal zeichnen. Ich kann Dinge hinzuf�gen und Dinge wieder wegl�schen. 
\end{drama}




\end{document}
