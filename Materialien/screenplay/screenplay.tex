\documentclass[a4paper,showtrims,11pt]{memoir}

\usepackage{dramatist}
\usepackage[latin1]{inputenc}


\pagestyle{plain}


\begin{document}



\Character[PERSON A, Person A]{Robert}{robert}
\Character[PERSON B, Person B]{Niko}{niko}
\Character[PERSON B, Person B]{Bernhard}{bernhard}


\scene[. -- Meta-Info vor der eigentlichen Pr�sentation]

\StageDir{
\begin{center}
Robert pr�sentiert die Gegebenheiten und Rahmenbedingungen
\end{center}
}

\begin{drama}
\robertspeaks 
\begin{itemize}
 \item Zielpublikum
  \begin{itemize}
    \item Analphabeten
    \item Nicht 100\%ig. Lesen einfacher W�rter m�glich. Oft via ``Musterekennung'', z.B. beim Einkaufen.
    \item Vermutlich auch einige Migranten $\rightarrow$ langsam sprechen, keine Fremdw�rter oder Anglizismen, Fokus auf Bilder, statt W�rter.
    \item Kurs wird im Rahmen einer Schreib/Lese-Ausbildung durchgef�hrt auf freiwilliger Basis.
  \end{itemize}
 \item Setting
  \begin{itemize}
    \item Freifach (Schulung) im Rahmen einer Schreib/Lese-Ausblidung f�r Menschen mit Schreib/Leseschw�che
    \item Teilnehmer arbeiten paarweise an Computern und k�nnen sich so gegenseitig helfen
  \end{itemize}
 \item Status
  \begin{itemize}
    \item Wir befinden uns in der zweiten oder dritten Einheit. 
    \item Basiswissen und -konzepte wurden vermittelt, Werkzeuge (Pinsel...)wurden bereits durchgemacht
  \end{itemize} 
\end{itemize}
\end{drama} % Meta-Information, setting, Zielpublikum...
\scene[. -- Begr��ung \& Einf�hrung]

\StageDir{
\begin{center}
Prezi anzeigen, Folie Nummer 1
\end{center}
}

\begin{drama}
\bernhardspeaks Ich darf euch ganz herzlich zum zweiten Vortrag der Vortragsreihe zum Thema Gimp begr��en. F�r all jene, die beim ersten Vortrag nicht anwesend waren, m�chte ich an dieser Stelle nochmals erw�hnen, dass diese Vortragsreihe parallel zum Grundkurs abgehalten wird. Selbstverst�ndlich sind wir uns dar�ber im Klaren, dass der Grundkurs noch nicht sehr weit fortgeschritten ist und hier vielleicht W�rter verwendet werden, die etwas komplizierter sind und im Grundkurs noch nicht behandelt wurden. Falls ihr W�rter nicht versteht, dann habt keine Scheu und unterbrecht uns einfach ganz kurz, das ist kein Problem. Obwohl in Gimp viele Funktionen durch sch�ne kleine Bildchen veranschaulicht werden, gibt es ein paar Funktionen die man nur durch das Men� erreichen kann. Am Anfang kann es vorkommen, dass ihr hier einzelne Funktionen vertauscht, da sich die W�rter mitunter doch sehr �hneln. Das ist v�llig normal und kein Problem, doch mit voranschreiten des Grundkurses wird euch die Bedienung von Gimp immer einfacher fallen. \newline \newline Soweit zu unserem kleinen Beispiel. Und damit wieder zu Niko.
\end{drama}

\scene[. -- R�ckblick auf den Inhalt des ersten Vortrages]


\StageDir{
\begin{center}
Folie mit den Werkzeugen
\end{center} 
}

\begin{drama}
\bernhardspeaks Wie sich sicher noch einige von euch erinnern k�nnen, haben wir uns im ersten Vortrag mit der Oberfl�che von Gimp vertraut gemacht und uns ein bisschen mit den unterschiedlichen Werkzeugen auseinander gesetzt. F�r all jene von euch, die beim letzten Mal nicht anwesend waren und denen der Begriff Gimp jetzt nichts sagt: Gimp kann man im Prinzip mit Photoshop vergleichen, das sicherlich vielen von euch ein Begriff ist. Im Gegensatz zu Photoshop muss man jedoch f�r Gimp nichts zahlen und kann es gratis aus dem Internet herunterladen. Das ist auch einer der Gr�nde, warum wir uns in dieser Vortragsreihe auf Gimp konzentrieren. \newline \newline Wie bereits erw�hnt, haben wir uns das letzte Mal mit den Werkzeugen auseinandergesetzt. Wir haben unter anderem gelernt, wie man Dinge ausschneiden und wieder einf�gen kann sowie wie man mit den Zeichenwerkzeugen in Gimp umgeht. Wie ihr euch sicher noch daran erinnern k�nnt, haben wir dabei immer auf einer Art "`virtuellem Papier"' gearbeitet. Heute werden wir einen Schritt weiter gehen und euch das Konzept der Ebenen vorstellen. Wir werden zu Beginn einmal kl�ren was denn Ebenen eigentlich sind, wof�r man sie in Gimp verwendet und worin der gro�e Vorteil von Ebenen liegt. Anschlie�end werden wir anhand ein paar einfacher Beispiele zeigen, wie man mit den Werkzeugen auf unterschiedlichen Ebenen arbeiten kann. Und damit geht's auch schon los. Bitte, Niko.
\end{drama} %Begruessung & Einfuehrung
\scene[. -- Live Demo mit gimp]

\StageDir{
\begin{center}
Prezi-Demo ist beendet. Umschalten auf den Gimp. Niko setzt sich an den Computer, Robert betritt ``die B�hne''.
\end{center}
}

\begin{drama}
\robertspeaks 
Wir haben jetzt auf den Gimp - den ihr ja schon kennt - umgeschaltet. Gehen wir noch einmal gemeinsam durch, was wir da sehen:\\
Links sehen wir die bekannte Werkzeugleiste mit den verschiedenen Werkzeugen zum zeichnen, ver�ndern und markieren von Bildelementen oder Bildteilen. 
Einige dieser Werkzeuge werden wir auch heute verwenden. Mehr dazu aber sp�ter. In der Mitte sehen wir das zu bearbeitende Bild Es ist unser Gimp-Maskottchen in einer Landschaft mit Sonne, Himmel, gr�nem Gras, einem Baum und �hnlichem und da sehen wir noch einen K�fig.\\
Rechts ist ist der Ebenendialog zu sehen. Und da unsere heutige Einheit sich um die Ebenen dreht, schauen wir uns gleich die mal n�her an.\\
Davor aber: gibt es Fragen?\\
Wir haben ja schon geh�rt, wie Ebenen funktionieren - wie Overheadfolien, transparente also durchsichtige Folien, die �bereinander gelegt werden. \StageDir{Zeige die Folien nochmal} Unsere Folien lassen das Licht durchscheinen, die Ebenen im Gimp sind prinzipiell so konzipiert, dass sie alles darunter Liegende v�llig abdecken k�nnen. So wie bei unserem Beispiel. Wir haben hier verschiedene Ebenen, die alle f�r sich nur einen Teil des Gesamtbildes beinhalten, alles andere ist auf den jeweiligen Ebenen, also Folien, transparent, so wie auf den Overhead Folien, die wir schon gesehen haben. Transparenz, also ``Durchsichtigkeit'' wird in Gimp �brigens mit diesem Hell/Dunkelgrauen Schachbrettmuster angezeigt.\\
Tats�chlich ist es so, dass wir die Ebenen im ebenendialog von oben nach unten sehen. Die oberste Ebene verdeckt also alle darunter liegenden, die zweite verdeckt alle weiteren und so weiter.
\StageDir{Auge und Folien ausblenden}
Wenden wir uns nun dem Bild im Zentrum zu. Es zeigt eine Landschaft mit einem Himmel, dem Gimp und einen K�fig. So wie wir auch einzelne Folien wegnehmen und wieder dazu geben k�nnen, k�nnen wir auch im Gimp Ebenen ``verstecken'' aber ohne sie zu l�schen. Dazu verwenden wir das Auge-Symbol. Ist das Auge im Ebenendialog sichtbar, ist auch die Ebene sichtbar. Auf diese Weise k�nnen wir z.B. den Gimp ``befreien'', indem wir auf das Augensymbol neben dem den K�fig klicken. Wir k�nnen aber auch z.B. unterschiedliche Hintergr�nde ins Bild bringen. Wir haben hier zwei Hintergr�nde vorbereitet: Gras und W�ste. Was glaubt ihr: was passiert, wenn ich auf das Auge neben dem Gras, das aktuell sichtbar ist, klicke? \StageDir{Frage an das Auditorium}
Eben: es verschwindet. Und wenn man stattdessen das Auge neben der W�ste klickt, kann man den Hintergrund einfach austauschen. Ohne Ebenen, wenn man den Hintergrund einfach gezeichnet h�tte, w�re das nicht m�glich gewesen.\\
Jetzt machen wir das wieder r�ckg�ngig. StageDir{W�ste wegklicken, Gras wieder an}
Wenn man aber mehrere Eben hat, an welcher arbeitet man gerade? \StageDir{Frage ins Publikum}
Es ist die Eben, die gerade markiert ist. Das erkennt man an der Markierung im Ebenendialog. Auch wenn die anderen Eben sichtbar sind bearbeitet man immer die Ebene. die markiert ist. Das kann zu seltsamen Ph�nomenen f�hren, da dar�ber liegende Ebenen die �nderungen verdecken k�nnen. Wir schauen uns das jetzt mal an. Niko: Bitte zeichne doch mal irgendwas in den Himmel. \StageDir{Markieren des Himmels, w�hrend Gras noch sichtbar ist, mit Pinsel (auch) im versteckten Bereich zeichnen, Ausblenden von Gras}. 
Wie man sieht, - blah - blah... \\ �brigens: Aufpassen! Das ist ein h�ufig gemachter Fehler im Gimp, dass man auf der falschen Ebene zeichnet!\\
Welche Ebenen-Manipulationen gibt es noch? Man kann z.B. eine Ebene rauf oder runter schieben. Damit ver�ndert sich die Reihenfolge der Ebenen und damit nat�rlich auch, was wir sehen. \StageDir{K�figebene hinter den Gimp legen} Kann mir jemand sagen warum der Gimp jetzt ``befreit'' ist?\\
Dies kann man auf verschiedene Art und Weise machen. Entweder mittels den zwei Pfeilen, die wir schon in der Pr�sentation gesehen haben, oder einfach indem wir eine Ebene mit der Maus anklicken und direkt verschieben.
Auf die selbe Weise ist es m�glich den Gimp z.B. halb hinter der Landschaft verschwinden zu lassen.\StageDir{Gimp hinter Gras legen - herumspielen, verschieben u.s.w.}
Es gibt noch mehr Operationen, die wir mit den Ebenen durchf�hren k�nnen.
\begin{itemize}
 \item neue Ebene\\Dialog zu ``Neue Ebene'' erkl�ren $\rightarrow$ letzter Punkt im Dialog ist transparenter Hintergrund
 \item Ebene verschieben\StageDir{Gimp selektieren und verschieben}
 \item Ebene verdoppeln
 \item Ebene verankern $\rightarrow$ sehen wir sp�ter
 \item Ebene l�schen
\end{itemize}
\StageDir{Rekapitulation: Was haben wir bis jetzt geh�rt bzw. gelernt, ebenen (Folien), Operationen darauf...}
\StageDir{Gibt es Fragen?}
Wenn wir uns das Bild mit dem eingesperrten Gimp ansehen sehen wir, dass die Pinselspitze unten unter dem K�fig raus schaut. Das passt so eigentlich gar nicht und sieht unrealistisch aus - oder? Wir werden nun versuchen, dies zu verbessern.Ich �bergebe jetzt an Bernhard. Bernhard: was kann man da machen?

\end{drama} % 1. Teil der Live demo
\scene[. -- Live Demo (Pinsel vor K�fig stellen)]

\StageDir{
\begin{center}
Robert stellt Frage ans Publikum: "`Wie ihr seht, ragt die Spitze des Pinsels unter dem K�fig hindurch. Wir w�rden nun gerne das Ganze so �ndern, dass die Spitze durch die Gitterst�be schaut. Wie k�nnen wir das machen?"' \newline \newline
Hier �bernehme dann ich.
\end{center}
}

\begin{drama}
\bernhardspeaks Wir wissen ja, dass wir Ebenen in Gimp rauf - und runterschieben k�nnen. Dadurch werden die Ebenen entweder weiter in den Vordergrund oder weiter in den Hintergrund verschoben. Wir k�nnten jetzt nun auf die Idee kommen die Ebene mit unserem Gimp Maskottchen in den Vordergrund zu verschieben, damit die Pinselspitze nicht mehr unterhalb des K�figs hindurchragt. Doch was wird dann passieren? \newline \newline Richtig, da sich das Maskottchen zusammen mit der Pinselspitze auf einer Ebene befindet, befreien wir das Maskottchen quasi aus dem K�fig, wenn wir die Ebenenreihenfolge �ndern. Die L�sung ist, dass wir die Pinselspitze ausschneiden und auf eine einzelne Ebene wieder einf�gen. Doch wie machen wir das am Besten? \newline \newline Wie ihr euch sicherlich noch erinnern k�nnt, gibt es das Zauberstab Werkezug, mit dem man recht einfach zusammenh�ngende Bereiche ausw�hlen kann, die �hnliche Farben haben. Wenn man mit dem Zauberstab Bereiche ausw�hlt, kann es vorkommen, dass es sinnvoll ist den Schwellwert zu erh�hen. Mit dem Schwellwert legt man fest um wieviel sich benachbarte Farben unterscheiden d�rfen, damit der Bereich als zusammenh�ngend erkannt wird. Nachdem die Pinselspitze ausgew�hlt ist, wird der ausgew�hlte Bereich mit Strg + X ausgeschnitten. Robert ist so nett und schreibt euch das Tastenk�rzel zum Ausschneiden auf die Tafel. \newline \newline Um die ausgeschnittene Pinselspitze wieder einzuf�gen, m�ssen wir eine neue Ebene erstellen. Als n�chstes m�sst ihr sicherstellen, dass die neu erstellte Ebene ausgew�hlt ist, bevor ihr die Pinselspitze wieder mit Strg + V einf�gt. Auch das Tastenk�rzel findet ihr auf der Tafel. Wenn man in Gimp etwas einf�gt, dann erstellt Gimp standardm��ig eine schwebende Auswahl. Die schwebende Auswahl kann man sich ein bisschen wie eine Zwischenebene vorstellen. Damit aus der schwebenden Auswahl eine richtige Ebene wird, muss die Auswahl verankert werden. Das geschieht mit dem Anker-Symbol. Wenn ihr die neue Ebene dann noch in den Vordergrund verschiebt, dann seht ihr wie die Pinselspitze nun aus dem K�fig schaut. Mit dem Augen-Symbol k�nnen wir die neue Ebene ein - und ausblenden und sehen, dass sich die Pinselspitze tats�chlich auf einer eigenen Ebene befindet. 
\end{drama} % Gimp Live Demo (2.Teil - Pinselspitze freistellen)







%%%%%%%%%%%%%%%%%%%%%%%%%%%%%%%%%%%%%%%%%%%%%%%%%%%%%%%%%%%%%%%%%%%%%%%%%%%%%%%%%%%%%%%%%%%%%%%%%%%%%%%%%%%%%%%%%%%

%\scene[. -- Vorwort]

%\StageDir{
%	\begin{center}
%		Keine Folie
%	\end{center}
%}

%\begin{drama}
%\aspeaks Bevor wir mit dem eigentlichen Vortrag starten, noch ein paar erkl�rende Worte zu unserer Zielgruppe. Der Vortrag richtet sich im Wesentlichen an Analphabeten, die sich mit dem Lesen und Schreiben schwer tun. Da der gr��te Teil der Analphabeten in �sterreich zur Gruppe der funktionalen Analphabeten geh�rt - einfache W�rter (z.B der eigene Name) k�nnen geschrieben und kurze S�tze gelesen werden, aber der Sinn l�ngerer Texte wird oftmals nicht verstanden - richtet sich der Vortrag verst�rkt an diese Personen. Das Bemerkenswerte ist, dass viele Analphabeten ihre Schreib - und Leseschw�che sehr gut vor anderen verbergen k�nnen, indem sie zum Beispiel Dinge, die andere aufschreiben w�rden, einfach ausw�ndig lernen (hier k�nnte man die Geschichte der Frau erz�hlen, die am Serviceschalter Auskunft �ber Zugverbindungen gibt und dabei alle Ankunfts - und Abfahrtszeiten auswendig gelernt hat). \newline Soviel zu unserer Zielgruppe. Wir hoffen, dass f�r euch der nachfolgende Vortrag nun etwas verst�ndlicher ist. 
%\end{drama}







%\scene[. -- R�ckblick auf den Inhalt des ersten Vortrages]


%\StageDir{
%\begin{center}
%Folie mit den Werkzeugen
%\end{center} 
%}

%\begin{drama}
%\aspeaks Wie sich sicher noch einige von euch erinnern k�nnen, haben wir uns im ersten Vortrag mit der Oberfl�che von Gimp vertraut gemacht und uns ein bisschen mit den unterschiedlichen Werkzeugen auseinander gesetzt. Unser heutiger Fokus liegt auf den Ebenen. Wir werden zu Beginn einmal kl�ren was denn Ebenen eigentlich sind, wof�r man sie in Gimp verwendet und worin der gro�e Vorteil liegt. Anschlie�end werden wir anhand ein paar einfacher Beispiele zeigen, wie man mit den Werkzeugen auf unterschiedlichen Ebenen arbeiten kann.  
%\end{drama}


%\scene[. -- Ebenen]


%\StageDir{
%\begin{center}
%Folie ????
%\end{center} 
%}

%\begin{drama}
%\bspeaks Nun zu den Ebenen. \textit{Hat von euch vielleicht jemand eine Vorstellung was man sich unter Ebenen vorstellen kann?} Eine Ebene kann man sich im Prinzip wie ein Blatt Papier oder eben diese Tafel vorstellen. An der Tafel kann ich ganz normal zeichnen. Ich kann Dinge hinzuf�gen und Dinge wieder wegl�schen. 
%\end{drama}

%%%%%%%%%%%%%%%%%%%%%%%%%%%%%%%%%%%%%%%%%%%%%%%%%%%%%%%%%%%%%%%%%%%%%%%%%%%%%%%%%%%%%%%%%%%%%%%%%%%%%%%%%%%%%%%%%%%


\end{document}
